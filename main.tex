\documentclass[12pt]{article}
 
\usepackage[margin=1in]{geometry} 
\usepackage{amsmath,amsthm,amssymb,graphicx,mathtools,tikz,hyperref,yfonts,tikz-cd, subfiles, enumitem, extarrows}
\usetikzlibrary{positioning}

\newtheorem{theorem}{Theorem}[subsection]
\newtheorem{corollary}{Corollary}[theorem]
\newtheorem{lemma}{Lemma}[theorem]
\newtheorem{lemma*}{Lemma}[subsection]
\newtheorem{proposition}{Proposition}[subsection]

\theoremstyle{definition}
\newtheorem{definition}{Definition}[subsection]

\theoremstyle{definition}
\newtheorem*{example}{Example}

\theoremstyle{definition}
\newtheorem*{examples}{Examples}

\theoremstyle{definition}
\newtheorem*{remark}{Remark}

\theoremstyle{definition}
\newtheorem*{remarks}{Remarks}

\theoremstyle{definition}
\newtheorem*{convention}{Convention}

\theoremstyle{definition}
\newtheorem*{conventions}{Conventions}

\DeclareMathOperator{\spec}{Spec}
\DeclareMathOperator{\Frac}{Frac}
\DeclareMathOperator{\nil}{nil}
\DeclareMathOperator{\rad}{rad}
\DeclareMathOperator{\ann}{Ann}
\DeclareMathOperator{\cok}{cok}
\DeclareMathOperator{\im}{im}
\DeclareMathOperator{\Hom}{Hom}
\DeclareMathOperator{\id}{id}
\DeclareMathOperator{\tor}{Tor}
\DeclareMathOperator{\ext}{Ext}
\DeclareMathOperator{\codim}{codim}
\DeclareMathOperator{\trdeg}{tr deg}
\DeclareMathOperator{\tot}{Tot}
\DeclareMathOperator{\qf}{qf}
\DeclareMathOperator{\mor}{Mor}
\DeclareMathOperator*{\colim}{colim}
\newcommand{\R}{\mathbb R}
\newcommand{\C}{\mathbb C}
\newcommand{\Z}{\mathbb Z}
\newcommand{\N}{\mathbb N}
\newcommand{\Q}{\mathbb Q}
\newcommand{\rp}{\R \mathbb P}
\newcommand{\cp}{\C \mathbb P}

\newcommand{\transverse}{\mathrel{\text{\tpitchfork}}}
\makeatletter
\newcommand{\tpitchfork}{%
  \vbox{
    \baselineskip\z@skip
    \lineskip-.52ex
    \lineskiplimit\maxdimen
    \m@th
    \ialign{##\crcr\hidewidth\smash{$-$}\hidewidth\crcr$\pitchfork$\crcr}
  }%
}
\makeatother

\newcommand{\diff}[2]{\frac{\partial{#1}}{\partial{#2}}}
\newcommand{\parens}[1]{{\left(#1\right)}}
\newcommand{\bracket}[1]{{\left[#1\right]}}
\newcommand{\curly}[1]{{\left\{#1\right\}}}
\newcommand{\pipe}[1]{{\left|#1\right|}}

\hypersetup{
    colorlinks=true,
    linkcolor=blue,
    filecolor=magenta,      
    urlcolor=cyan,
}

\begin{document}
\date{}

 
\title{Math 225B Lecture Notes\\
Prof. Ko Honda\\
Winter 2020}
\maketitle

\tableofcontents{}

\newpage

\section{Lecture 1 - 1/6}
\subfile{lectures/lecture01}
\newpage

\section{Lecture 2 - 1/8}
\subfile{lectures/lecture02}
\newpage

\section{Lecture 3 - 1/10}
\subfile{lectures/lecture03}
\newpage

\section{Lecture 4 - 1/13}
\subfile{lectures/lecture04}
\newpage

\section{Lecture 5 - 1/15}
\subfile{lectures/lecture05}
\newpage

\section{Lecture 6 - 1/17}
\subfile{lectures/lecture06}
\newpage

\section{Lecture 7 - 1/22}
\subfile{lectures/lecture07}
\newpage

\section{Lecture 8 - 1/24}
\subfile{lectures/lecture08}
\newpage

\section{Lecture 9 - 1/27}
\subfile{lectures/lecture09}
\newpage

\section{Lecture 10 - 1/29}
\subfile{lectures/lecture10}
\newpage

\section{Lecture 11 - 2/3}
\subfile{lectures/lecture11}
\newpage

\section{Lecture 12 - 2/5}
\subfile{lectures/lecture12}
\newpage

\section{Lecture 13 - 2/12}
\subfile{lectures/lecture13}
\newpage

\section{Lecture 14 - 2/13}
\subfile{lectures/lecture14}
\newpage

\section{Lecture 15 - 2/14}
\subfile{lectures/lecture15}
\newpage

\section{Lecture 16 - 2/19}
\subfile{lectures/lecture16}
\newpage

\section{Lecture 17 - 2/20}
\subfile{lectures/lecture17}
\newpage

\section{Lecture 18 - 2/21}
\subfile{lectures/lecture18}
\newpage

\section{Lecture 19 - 2/24}
\subfile{lectures/lecture19}
\newpage

\section{Lecture 20 - 2/26}
\subfile{lectures/lecture20}
\newpage

\section{Lecture 21 - 2/28}
\subfile{lectures/lecture21}
\newpage

\section{Lecture 22 - 3/2}
\subfile{lectures/lecture22}
\newpage

\end{document}