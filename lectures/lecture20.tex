\subsection{Poincar\'e Duality}

\subsubsection{Poincar\'e Lemma cont.}
We begin by continuing our proof of the Poincar\'e Lemma from the previous lecture.

\begin{proof}
    Note that from here on, we shall abuse notation and write $\pi^* \phi = \phi$. This is not too heinous, as $\pi^* \phi$ is basically just $\phi$ extended constantly in the vertical direction.
    
    Recall that we were left to show that the $K$ we defined previously was indeed a chain homotopy operator between $id$ and $e_* \circ \pi_*$. Indeed, we see that $(id - e_* \pi_*)\phi f = \phi f$. On the other hand,
    \begin{align*}
        (dK - Kd)\phi f &= -Kd(\phi f)\\
        &= -K\parens{d(\phi \circ f) + (-1)^i \phi \diff{f}{x} dx + (-1)^i \phi \diff{f}{t} dt}\\
        &= (-1)^{i + 1} \phi\parens{\int_{-\infty}^t \diff{f}{\tau} d\tau - \int_\R \diff{f}{\tau} d\tau \int_{-\infty}^t e(\tau) d\tau}\\
        &= (-1)^{i + 1} \phi f
    \end{align*}
    so $id - e_* \circ \pi_* - (-1)^{i + 1}(dK - Kd)$ and we have the desired isomorphism on cohomology.
\end{proof}

We now state a key example of the Poincar\'e Lemma. Indeed, this allows us to compute
\[
    H_c^i(\R^n) = H_c^{i - 1}(\R^{n - 1}) = \dots = H_c^0(\R^{n - i}) =
    \begin{cases}
        0 & n > i\\
        \R & n = i
    \end{cases}
\]
Of course, for $i > n$ we of course have $H_c^i(\R^n) = 0$. This reveals as interesting pattern

\[
    \begin{tabular}{c|c|c}
                    & $H_c^i(\R^n)$ & $H^i(\R^n)$\\
    \hline
        $i = n$     & $\R$          & 0\\
    \hline
        $i = n - 1$ & 0             & 0\\
    \hline
        $\vdots$    & \vdots        & \vdots\\
    \hline
        $i = 0$     & 0             & $\R$\\
    \end{tabular}
\]
so we'd suspect to have $H^i(M) = H_c^{n - i}(M)$ for $n = \dim M$.

\subsubsection{Poincar\'e Duality}

\begin{definition}
    An open cover of a manifold is said to be good if all nonempty finite intersections are diffeomorphic to $\R^n$. It is a fact, which we state without proof, that all manifolds admit a good cover.
\end{definition}

\begin{lemma*}
    If $M$ has a finite good cover then $H^i(M)$ and $H_c^i(M)$ are finite dimensional.
\end{lemma*}
\begin{proof}
    Homework.
\end{proof}

\begin{theorem}[Poincar\'e Duality]
    If $M^n$ is oriented and has a finite good cover, then the pairing
    \begin{align*}
        \int: H^i(M) \otimes H_c^{n - i}(M) &\longrightarrow \R\\
        \omega \otimes \eta \mapsto \int_M \omega \wedge \eta
    \end{align*}
    is nondegenerate. Equivalently, the associated map $H^i(M) \longrightarrow (H_c^{n - i}(M))^*$ is an isomorphism.
\end{theorem}
\begin{corollary}
    If $M^n$ is oriented and compact, $H^i(M) \cong (H^{n - i}(M))*$
\end{corollary}
\begin{proof}
    Poincar\'e duality is a local statement which has been globalized.
    
    We first show that result for $M = \R^n$. Indeed, for the map $H^i(\R^n) \longrightarrow (H_c^{n - i}(\R^n))^*$, the only nontrivial case is $i = 0$. Indeed, for $i = 0$, the map takes $1 \mapsto \parens{\omega \mapsto \int_{\R^n} \omega}$. Obviously, forms with $\int_{\R^n} \omega \neq 0$ exist, so by dimension counting this is an isomorphism.
    
    Now, suppose we are in the general case and have a finite good cover $\{U_1, \dots, U_k\}$. Then by the above argument, Poincar\'e duality holds for each $U_i \cong \R^n$ and all finite nonempty intersections $U_{i_1} \cap \dots \cap U_{i_l}$.
    
    [Proof continued next time.]
\end{proof}